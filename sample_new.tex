%%%%%%%%%%%%%%%%%%%%%%%%%%%%%%%%%%%%%%%%%%%%%%%%%%%%%%%%%%%%%%%%%%%%%%%%%%%%%%%%
%2345678901234567890123456789012345678901234567890123456789012345678901234567890
%        1         2         3         4         5         6         7         8

\documentclass[letterpaper, 10 pt, conference]{ieeeconf}  % Comment this line out
                                                          % if you need a4paper
%\documentclass[a4paper, 10pt, conference]{ieeeconf}      % Use this line for a4
                                                          % paper

\IEEEoverridecommandlockouts                              % This command is only
                                                          % needed if you want to
                                                          % use the \thanks command
\overrideIEEEmargins
% See the \addtolength command later in the file to balance the column lengths
% on the last page of the document



\usepackage{draftwatermark}
% The following packages can be found on http:\\www.ctan.org
%\usepackage{graphics} % for pdf, bitmapped graphics files
%\usepackage{epsfig} % for postscript graphics files
%\usepackage{mathptmx} % assumes new font selection scheme installed
%\usepackage{times} % assumes new font selection scheme installed
%\usepackage{amsmath} % assumes amsmath package installed
%\usepackage{amssymb}  % assumes amsmath package installed

\title{\LARGE \bf
Interception of moving objects with a robotic arm in a simulated environment
}

\author{Juan Garcia, Harrison Jones and Arash Rouhani*
  \thanks{\texttt{*\{jgarcia39,harrisonhjones,rarash\}@gatech.edu}}
}


\begin{document}

\maketitle
\thispagestyle{empty}
\pagestyle{empty}


%%%%%%%%%%%%%%%%%%%%%%%%%%%%%%%%%%%%%%%%%%%%%%%%%%%%%%%%%%%%%%%%%%%%%%%%%%%%%%%%
\begin{abstract}

The issues of catching a moving projectile can be divided into two sets,
the first set of issues is to perceive the object and to estimate its
future path.  The second set of issues is to knowing a trajectory find
the best stance to catch the object in. In this paper, we focus mainly
on the second issue but we also insert simple models of vision
distortion and path regression to complete the model of the application.
The approach to select one out of multiple robot arm stances we take is
multi-goal path planning, we introduce a modified version of
Rapidly-Exploring Random Trees (RRTs) that can grow to multiple goals.
We benchmarked our Multi-goal RRT in a simulated environment but got
beaten by the regular RRT.

\end{abstract}


%%%%%%%%%%%%%%%%%%%%%%%%%%%%%%%%%%%%%%%%%%%%%%%%%%%%%%%%%%%%%%%%%%%%%%%%%%%%%%%%
\section{INTRODUCTION}

\subsection{Motivation}

The motivation for this project came from two sources. One is a video which
illustrates a robot capable of "catching" a piece of trash thrown using a
vision system and limited path planning. The little trash robot is pretty
interesting, and useful, and initially the thought of expanding on the idea was
toyed with. The other source of motivation came from a previous piece of work
performed under Professor Mike Stillman at Georgia Tech. Previously a group
under Stillman had worked on an algorithm which blocked known sword path,s in a
simulated environment. The group had not explored the idea of predicting a
sword swing given the current path of the sword.

\subsection{Problem}

These two interesting ideas were combined into a "general purpose" problem
which covered both motivating sources. The problem therefore tackled in this
project is how to intercept a moving object using a robot arm in a noisy
simulated environment given limited information on that path of the object.
This problem is interesting because it is fairly general purpose covers a great
number of practical applications such as catching balls, blocking punches,
military applications, etc.

\subsection{Solution}

The solution described in this paper utilizes an algorithm which tracks the
objects, predicts its future path, and then moves a virtual robot arm to
intercept the object. This is done by using two different prediction
algorithms, a linear predictor, used for punching, handshakes, etc, and a
quadratic predictor, used for projectile objects such as trash or balls. After
computing a predicted path a multi-goal RRT is used to move the robot arm to
intercept the object while still avoiding obstacles in its environment.



\section{RELATED WORK}

An extension of RRTs for multi goal purposes is introduced here. There are also
other approaches to solving multi-goal motion planning (add referencess). As
for the modeling of the projectile motion, nothing is used from academia.

\section{METHODS}

We'll cover the methods in the order they are used in a complete simulation.
First, we must generate a projectile path, then distortion is applied. In order
to know where it's possible to catch the object some path prediction is used.
Then to actually get a series of possible robot stances, inverse kinematics is
applied to get some corresponding joint configuration for each projectile
position. Finally, and most highlighted part in this paper, we search for the
easiest to reach joint space configuration. It should be noted that this whole
process is done iteratively, at fixed time steps we replan taking in account
the most recent observation of the projectile.

\subsection{Path Projection}

We basically use a discretized version of:

\[
  x(t) = vt + at^2
\]

where we set $a = -g$ where $g$ is some gravity constant.

\subsubsection{Distortion}

We basically use a discretized version of:

\[
  x_{observed}(t) = x(t) + noise
\]

Some sort of wind is also an extension to this, but is modeled by setting to
$a$ to $-g+c$ where $c$ is constant.

\subsection{Path Prediction}

Given some sample points, we look at the last ones and work backwards from two
assumed equations, one that models linear motion and the other models
projectile motion. The equation

\[
  x_{observed}(t) = vt + at^2
\]

will cover both cases, only that you set $a=0$ in the linear version.

Note that this is of course not exact as it does not compensate for the unknown
noise, in order to compensate for the noise one can look at chunks of points as
one point and work backwards from that.

\subsection{Inverse kinematics}

Knowing the trajectory, one can use inverse kinematics to find a corresponding
joint space configuration. We do use the pseudo inverse jacobian method.

\subsubsection{Distance in joint space}

We defined distance as the infinity norm.

\subsection{Planning}

We compare the results using a single goal rrt and a multi goal rrt. The multi
goal rrt is a regular rrt only that it yields the first path that leads to any
of the goals. When it does a greedy expansion to a goal, it randomly picks the
goal to grow towards.

\subsection{Replanning}

With a fixed $\Delta t$ we keep iterating and do replanning as more information
of the trajectory gets available all the time.

\section{EXPERIMENTS}

Our test environment consisted of a 7 joint robot with only a few objects around it. All of the objects were on the ground. The object was always shot from X meters away towards the hand.

In this setting single goal rrt won.

\section{ANALYSIS}

Multi goal rrt might have been changing its mind all the time between each replanning making it distracted.

\section{DISCUSSION}

It would be interesting to test the algorithms in an environment where there
are more objects that the arm can't touch. Also, the multi goal rrt does not
need to stop once it reaches one goal, it could instead keep expanding until it
hits all goals and then pick the one with the shortest path.

As an extra benefit to both rrt planners, path shortening could be applied.

\begin{thebibliography}{99}

\bibitem{c1} TODO: Add real references

\end{thebibliography}




\end{document}
